\anonsection{Введение}

Наноразмерные устройства со всё большей скоростью интегрируются в жизнь современного человека в связи с набиращей обороты тенденцией к миниатюризации электронных устройств. Один из наглядных примеров - устройства на основе пористого оксида алюминия (ПОА). За последнее десятилетие было проведено множество работ, связанных с синтезом этого материала и возможностями его использования. Однако, как и со многими наноразмерными материалами, несмотря на разнообразие возможностей и наличия огромного количества фундаметальных данных, одной из основных проблем остается дороговизна исходных материалов и процесса изготовления.

Упорядоченные структуры (массивы нанонитей, нанотрубки, наноточки) представляют научный и прикладной интерес как для понимания оптических, магнитных, электрических, механических и тепловых свойств наноматериалов, так и для создания на их основе приборов с новыми физическими характеристиками \cite{belov-osobennosti}.

Процесс анодирования является одним из наиболее распространённых методов синтеза ПОА с периодическим расположением пор, ввиду технологической простоты и эффективности. 
Несмотря на широкое распространение данной технологии, лишь в последние годы она стала активно модернизироваться в связи с возможностью применения для получения ПОА с определённой морфологией и, как следствие, с заранее установленными характеристиками. Основой для пористых оксидных плёнок могут служить такие материалы, как $Ti$, $Si$, $InP$, $Nb$, $Sn$, $Al$ и т.д. $Al$ является наиболее перспективным из них, поскольку ПОА имеет наноразмерную гексагональную пористую структуру, а также обладает высокой механической прочностью, уникальными оптическими и диэлектрическими свойствами. Синтез ПОА с широким набором структурно-морфологических и электрофизических характеристик возможен благодаря изменению условий анодирования.

На сегодняшний день пленки ПОА находят широкое применение при использовании в качестве матриц для создания упорядоченных массивов анизотропных наноструктур разного рода состава.

Повышенное внимание к таким структурам обращено в связи с возможностью изучения на их примере фундаментальных задач (процессов самоорганизации и магнетизма в пространственно-упорядоченных наносистемах), а так же решения широкого круга прикладных проблем, связанных с созданием высокоэффективных гетерогенных катализаторов, гиперболических метаматериалов, а также получением магнитных нанокомпозитов для устройств хранения информации со сверхвысокой плотностью записи \cite{belov-osobennosti}.

\clearpage
