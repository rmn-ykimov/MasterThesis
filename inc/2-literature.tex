\section{Обзор литературных источников}

\subsection{Образование оксида на поверхности алюминия}

Поверхность металлического алюминия самопроизвольно окисляется на воздухе при нормальных условиях. Это объясняется большим отрицательным значением изменения свободной энергии Гиббса \cite{atraschenko-hyperbolyc}:

\begin{equation}
2Al_\text{(тв)} + \frac{3}{2}O_{2\text{(г)}}\to Al_2O_{3\text{(аморф)}};  \Delta G^0_{298}298=−1308\text{кДж/моль}
\end{equation}

При попадании воды на поверхность так же идёт процесс окисления:
\begin{equation}
2Al_\text{(тв)} + 3H_2O_\text{(ж)}\to Al_2O_{3\text{(аморф)}}+3H_2\text{(г)};  \Delta G^0_{298}298=−597 \text{кДж/моль}
\end{equation}

Плотный слой $Al_2O_3$, образующийся на поверхности $Al$ имеет небольшую толщину (порядка $5-20$нм). Он защищает металл от процессов дальнейшего окисления, а также от контакта с другими реактивами. Метод электрохимического анодного окисления (анодирование) $Al$ в водных растворах электролитов является другим способом получения оксидной плёнки является \cite{atraschenko-hyperbolyc}. В этом случае удаётся получить более толстый слой $Al_2O_3$.

Реакция на аноде (происходит образование оксида):

\begin{equation}
2Al_\text{(тв)} + \frac{3+n}{2}H_2O_\text{(ж)}–3e^-\to \frac{1}{2}Al_2O_3*(H_2O)_{n\text{(тв)}}+3H^+; n=0÷3    
\end{equation}

Реакция на катоде (восстанавливаются ионы водорода):

\begin{equation}
H^++e^-=\frac{1}{2}H_{2\text{(г)}}  
\end{equation}

Реакция, которая протекает на аноде ($Al$), зависит от таких факторов, как потенциал электрода, температуры и $pH$ среды, которые, в свою очередь, определяются типом используемого электролита.

В связи с изобилием полиморфных и гидратных форм оксида алюминия, анодный $Al_2O_3$ имеет переменный состав. Данные, полученные путем экспериментов, указывают на то, что анодный $Al_2O_3$ является рентгеноаморфным твёрдым веществом, состоящим из гидратированного оксида алюминия $Al_2O_3·(H_2O)_n$, где $n=0÷3$, развитая внутренняя поверхность которого адсорбирует анионы и катионы использованного электролита.

\subsection{Типы плёнок $Al_2O_3$}

В научной литературе делают акцент на $2$ основных типах оксида (рис. \ref{poa_scheme}).

\addimg{poa_scheme}{0.75}{Схематическое изображение плёнок $Al_2O_3$ \cite{napolsky-sintez}.}{poa_scheme}

В электролитах, не растворяющих оксид ($5 <pH <7$), в частности, в растворах борной кислоты, может быть синтезирован барьерный тип плёнок \cite{yakovleva-zakonomernosti}, \cite{viharev-osobennosti}. Пористый тип плёнок возникает в электролитах, которые слабо растворяют оксид (серная, фосфорная и щавелевая кислоты \cite{li_y}).

Оба этих вида (барьерный и пористый) состоят из внутреннего и внешнего слоя (рис. \ref{poa_scheme})). Внутренний слой является чистым $Al_2O_3$, тогда как внешний содержит различные примеси в виде ионов. В связи с высоким удельным сопротивлением плёнок ($10^{12}-10^{13} \text{Ом∙см}$ \cite{grilihes-pokrytiya}), задача получения слоёв ПОА большой толщины (порядка десятков микрометров) электрохимическим методом связана со значительными трудностями. Максимальная толщина пористого оксида алюминия при этом будет заметно больше ($100 \text{мкм}$ и более) по сравнению с толщиной плёнок барьерного типа в связи с тем, что она более проницаема для раствора электролита, хотя при увеличении толщины такой плёнки теряется упорядоченность наноканалов \cite{napolskiy-avtoref}.

Экспериментально доказано, что толщина оксидного слоя барьерного типа фактически полностью определяется величиной приложенного напряжения и слабо зависит от используемого электролита и температуры. Максимальная толщина плёнки оксида алюминия барьерного типа, достигнутая при напряжении $500 - 700 \text{В}$, составляет ~$1 \text{мкм}$.

Барьерный тип плёнок может быть получен в электролитах, не растворяющих оксид ($5 < pH < 7$), например, в растворах борной кислоты $H_3BO_3$, винной $C_4H_6O_6$ или лимонной $C_6H_8O_7$ \cite{varipaev-prakticum}. Плёнки пористого типа образуются в слабо растворяющих электролитах, таких как серная $H_2SO_4$  \cite{yang-influence}, фосфорная $H_3PO_4$ \cite{mombello},\cite{wang-analysis},\cite{coz-analysis} и щавелевая $H_2C_2O_4$ \cite{grigoryev-2d},\cite{jaafar-alumina} кислоты.

По литературным данным, анодный оксид алюминия является твёрдым рентгеноаморфным веществом, содержащим некоторое количество адсорбированной воды \cite{napolsky-sintez},\cite{petukhov} и представляющий собой смесь гидроксида алюминия $Al(OH)_3$, оксигидроксида алюминия $AlOOH$, гидратированного оксида алюминия $Al_2O_3*(H_2O)_{0..3}$ и инкорпорированных из раствора электролита анионов \cite{masuda-mosaic}, \cite{napolsky-samoorganizatsiya}.

Содержание анионов в структуре анодированного оксида алюминия сильно зависит от типа электролита и может составляет $2\%$ для $(COOH)_2$ до $13\%$ для $H_2SO_4$ \cite{sheasby-treatment}. 

Распределение анионов по толщине стенки поры и по высоте плёнки имеет сложный характер \cite{coz-analysis},\cite{patermarakis-investigation}. По данным некоторых авторов в барьерном слое анодного оксида алюминия присутствуют нанокристаллическая фаза $\gamma '-Al_2O_3$ с размерами кристаллитов от $2$ до $10 \text{нм}$. $\gamma '-Al_2O_3$ представляет собой промежуточную форму между аморфным оксидом алюминия и кристаллической фазой $\gamma-Al_2O_3$.

Кроме того было показано, что плёнки оксида алюминия как барьерного, так и пористого типов состоят из двух слоёв: внутреннего высокочистого оксида алюминия и внешнего слоя оксида алюминия, загрязнённого различными примесями, что показано на рисунке \ref{poa_scheme} \cite{napolsky-sintez}, \cite{petukhov}. Степень загрязнения зависит от типа электролита и концентрации адсорбированных анионов.

\subsection{Механизм образования пористой структуры}

Образование ПОА рассматривается на основе кинетики гальваностатического (при фиксированной плотности тока) и потенциостатического (при фиксированном напряжении) оксидирования \cite{atraschenko-hyperbolyc}.

На ранней стадии роста анодированного оксида алюминия на зависимостях напряжения от времени при гальваностатическом анодировании выделяют три участка, характерных для стадий образования пористого слоя. Стадии показаны на рис. \ref{galvanostatic_mode}.

\addimg{galvanostatic_mode}{0.6}{Зависимость напряжения при гальваностатическом режиме от времени анодирования \cite{atraschenko-hyperbolyc}.}{galvanostatic_mode}

Как видно из рис. \ref{galvanostatic_mode}, на участке 1 происходит рост напряжения, который оъясняется нарастанием на поверхности $Al$ барьерного слоя. Этот этап завершается при достижении максимально возможной толщины барьерного слоя при данных условиях, которая, к тому же, соответствует максимуму напряжения на временной диаграмме.

Затем, на следующей стадии, наблюдается спад напряжения, который возникает из-за зарождения пор в барьерной плёнке.

Заключительная стадия характеризуется почти постоянным во времени значением напряжения. В ходе неё происходит рост пористой части плёнки, в то время как толщина барьерного слоя остаётся практически постоянной. Необходимо подчеркнуть, что в установившемся гальваностатическом режиме при длительном анодировании происходит рост напряжения, обусловленный увеличением толщины пористой плёнки \cite{atraschenko-hyperbolyc}.

Тип плёнки, образующейся на аноде при потенциостатическом окислении, определяет вид хроноамперометрической кривой.

На первоначальном этапе анодирования при образовании плёнки барьерного и пористого типа оба графика идут одинаково. Однако плотность тока в случае плёнки барьерного типа экспоненциально спадает, и, в конечном счёте, плотность тока становиться равной вкладу ионной компоненты тока.

В случае образования плёнок пористого типа на первом этапе окисления плотность тока резко падает (участок 1 на рис.\ref{j-t}), затем на участке 2 проходит через минимум, затем резко возрастает и проходит через максимум (участок 3), после чего выходит на постоянное значение (участок 4). Можно разложить ток $j_p$, соответствующий образованию плёнки пористого типа, на две составляющие:
\begin{itemize}
    \item $j_b$ – плотность тока при образовании плёнки барьерного типа
    \item $j_{hp}$ – некоторая гипотетическая плотность тока, которая соответствует плотности тока, связанного с образованием пор.
\end{itemize}
Плотность тока $j_p$ определяется только приложенным потенциалом, в то время как $j_{hp}$ зависит от используемого электролита, температуры, при которой проводят окисление, и напряжения анодирования.

\addimg{j-t}{0.6}{Зависимость плотности тока от времени анодирования при постоянном напряжении \cite{napolsky-sintez}.}{j-t}

Механизм образования пор схематично показан на рис.\ref{poa_stages}.

Образование пор протекает в четыре стадии, соответствующие четырём участкам на зависимости тока от времени на рис.\ref{galvanostatic_mode}.

На первой стадии окисления, поверхность алюминия покрывается барьерным слоем, который состоит из непроводящего оксида алюминия. Напряжённость электрического поля резко возрастает в углублениях оксидной плёнки (стадия 2 на рис.\ref{poa_stages}, что приводит к протеканию процесса растворения оксида за счёт локального роста температуры (стадия 3 на рис.\ref{poa_stages}).

\addimg{poa_stages}{0.6}{Схематичное изображение стадий образования пористой структуры \cite{petukhov}}{poa_stages}

Ввиду конкуренции соседних точек стока заряда, часть пор прекращают свой рост, что приводит к некоторому уменьшению плотности тока на хроноамперометрической кривой (стадия 4, рис.\ref{poa_stages}). В конечном счёте $j_p$ выходит на постоянное значение, соответствующее равномерному росту пор. Некоторое уменьшение плотности тока в процессе длительного окисления связано с затруднённой диффузией ионов в порах анодного оксида алюминия \cite{petukhov}.

\subsection{Самоорганизация ПОА}

\addimg{poa_self-ordering}{0.75}{Просвечивающая электронная микроскопия пленок пористого оксида алюминия с самоупорядоченной структурой, полученных при а) $25\text{В}$ в $0,3\text{М}$ $H_2SO_4$, б) $40\text{В}$ в $0,3\text{М}$ $(COOH)_2$, в) $195\text{В}$ в $0,1\text{М}$ $H_3PO_4$ \cite{napolsky-sintez}.}{poa_self-ordering}

Структура пор плёнок имеет плотнейшую гексагональную упаковку, поры располагаются строго перпендикулярно поверхности плёнки \cite{evertsson}. Самоупорядочение происходит вследствие механических напряжений, которые вызванны силами отталкивания между соседними порами. Такакя структура формируется только при определенных условиях \cite{napolsky-sintez}. Например, оксид алюминия с расстоянием между порами равным $50$, $65$, $100$, $420$ и $500\text{нм}$ образуется при напряжении $19$ и $25\text{В}$ в серной кислоте, при $40\text{В}$ – в щавелевой, при $160$ и $195\text{В}$ – в фосфорной кислотах соответственно, что показано на рис.\ref{poa_self-ordering}. Расстояние между соседними порами $D_{int}$ зависит преимущественно от напряжения анодного окисления и оказывается прямо пропорционально ему с коэффициентом пропорциональности $k$, где $2.5≤k\text{(нм/В)}≤2.8$:

\begin{equation}
D_{int}=kU
\end{equation}

\addimg{d-v}{0.6}{Зависимость расстояния между соседними порами от потенциала анодирования \cite{petukhov}.}{d-v}

\addimg{two_staged}{0.75}{Двухстадийная методика получения оксида алюминия с высокоупорядоченной структурой пор \cite{napolsky-sintez}.}{two_staged}

Синтез плёнок, высокоупорядоченных по всему объему, возможен только в случае метода синтеза с двумя стадиями. При таком подходе плёнку, полученную на первой стадии, удаляют, а алюминиевую пластинку с микрорельефом подвергают повторному анодированию (рис.\ref{two_staged}). Таким образом удаётся получить ПОА с упорядоченной по всему объёму структурой.

В ходе первой стадии поверхность высокочистого $Al$ (не менее $99,99\%$) очищают при помощи ацетона, а также  подвергают травлению в смеси кислот $HF$/$HNO_3$/$HCl$. После, алюминиевую подложку отжигают в течение 3 часов при $500^{\circ}C$ для роста зёрен металлического алюминия. Увеличение размера зёрен, произошедшее в ходе рекристаллизационного отжига в исходной пластинке алюминия приводит к увеличению размеров доменов (областей упорядочения) в ПОА. Для уменьшения шероховатости поверхности $Al$ прибегают к электрохимической полировке в смеси, содержащей $\frac{1}{4}HClO_4 + \frac{3}{4}C_2H_5OH$. Полировка также способствуют получению упорядоченной пористой структуры оксида алюминия с большим размером доменов.

После предварительной подготовки поверхности проводится первое анодное окисление алюминия \cite{napolsky-sintez}.

\addimg{poa_sem}{1}{Данные сканирующей электронной микроскопии с пористой плёнки после первого анодного окисления при 195 В в 0,1 М $H_3PO_4$: а) поверхность оксидной плёнки; б) нижняя часть мембраны (микрофотография получена после селективного растворения алюминия) \cite{napolsky-sintez}.}{poa_sem}

На начальной стадии процесса образующиеся поры малоупорядочены (см. рисунки \ref{two_staged} в и \ref{poa_sem} а. Однако, в ходе первого длительного процесса окисления в результате действия сил отталкивания между соседними порами, происходит самоупорядочение пористой структуры. В результате этого на границе раздела оксид/металл образуется периодическая структура с плотнейшей гексагональной упаковкой пор в $Al_2O_3$ (см. рис.\ref{two_staged} б).

После первого анодного окисления плёнку $Al_2O_3$ растворяют в смеси $CrO_3/H_3PO_4$, не затрагивая слоя $Al$, чтобы получить реплику нижней части оксидной плёнки, имеющей упорядоченную структуру. В результате последующего (второго) анодного окисления при тех же условиях, что и при первом окислении, удаётся получить плёнку оксида алюминия с высокой степенью упорядочения пор. При необходимости поры можно равномерно расширить химическим травлением, например, в $0.5 – 1\text{М}$ фосфорной кислоте \cite{napolsky-sintez}, \cite{petukhov}.

\subsection{Параметры анодирования, влияющие на микроструктуру анодного оксида алюминия}

\subsubsection{Напряжение}

Величина используемого напряжения определяет толщину барьерного слоя, которая в случае образования упорядоченной структуры равна половине расстояния между центрами пор. Кроме того, за счёт варьирования напряжения в процессе анодирования, могут быть синтезированы мембраны анодного оксида алюминия, обладающие иерархической структурой пор \cite{sacco},  \cite{meng-nanowires}. При уменьшении напряжения в $\sqrt n$ происходит ветвление одной поры на $n$ пор, что показано на рис.\ref{nanotubes}.

\addimg{nanotubes}{0.45}{Микрофотография углеродных нанотрубок, синтезированных в поре разветвлённой на а) 2; b) 3; c) 4 и d) 16 пор \cite{suminov-modifitsirovanie}.}{nanotubes}

\subsubsection{pH электролита}

Для получения самоупорядоченной пористой структуры при фиксированном значении напряжения необходимо тщательно подбирать тип и концентрацию электролита. Обычно при небольших значениях напряжения ($5-40\text{В}$) анодирование алюминия проводят в растворах серной кислоты, при средних значениях ($30-120В$) применяют растворы щавелевой кислоты. При высоких ($80-200\text{В}$) – растворы фосфорной кислоты. Данные ограничения связаны с проводимостью, а также $рН$ используемого электролита. Например, если анодирование алюминия проводится в серной кислоте при высоком напряжении, очень часто наблюдается пробой оксидного слоя. $рН$ электролита определяет также размер пор, образующихся в процессе окисления. Чем меньше значение $рН$, тем меньший потенциал необходим для растворения оксида на нижней границе поры, и, соответственно, тем ниже диаметр формируемых пор. Таким образом, поры большого диаметра образуются в фосфорной кислоте, в то время как использование серной кислоты приводит к образованию пор малого диаметра \cite{napolsky-sintez}.

\subsubsection{Температура}

В процессе анодного окисления алюминия необходимо поддерживать как можно более низкую температуру электролита для предотвращения растворения образующегося оксида в кислой среде. Например, анодирование при $40\text{В}$ в щавелевой кислоте проводят при температуре $0-18^{\circ}С$, а окисление при $195\text{В}$ – в фосфорной кислоте при $0-2°С$. Ещё одним фактором, требующим охлаждения электролита, является предотвращение локального перегрева на дне пор в течение анодирования (особенно при окислении при высоких напряжениях). Локальный перегрев приводит к неравномерному распределению электрического поля на дне пор, что часто приводит к электрическому пробою оксидного слоя. При отсутствии контроля температуры в процессе анодирования образуются трещины и разрывы в плёнке пористого оксида алюминия, изображённые на рис.\ref{overheat}.

С другой стороны, при очень низких температурах возможно частичное замерзание электролита, что приведет к  изменению концентрации кислоты. Кроме того, температура во многом определяет скорость роста оксидного слоя: чем ниже температура, тем меньше скорость роста пор \cite{petukhov}.

\addimg{overheat}{0.6}{Пробой плёнки анодного оксида алюминия в результате локального перегрева \cite{petukhov}.}{overheat}

\subsubsection{Влияние примесей}

Для получения упорядоченной структуры пор требуется использовать высокочистый алюминий ($>99,99\%$). Окисление алюминия, содержащего большее количество примесей, приводит к формированию неупорядоченной структуры из-за различий в коэффициентах объемного расширенияразличных оксидов и наличия дефектов в исходном металле и растущей пленке.

\subsubsection{Прочие факторы}

Необходимо выделить ряд дополнительных факторов, влияющих на структуру пористого слоя: размер зёрен алюминия, подвергаемого окислению; шероховатость поверхности; продолжительность первого окисления. Способы, позволяющие минимизировать влияние этих дефектов:

\begin{itemize}
    \item неровности на поверхности $Al$ удаляют механической или электрохимической полировкой.
    \item для удаления пузырьков кислорода, уменьшения локального перегрева поверхности, а также гомогенной диффузии анионов в каналы пор в процессе анодирования необходимо интенсивное перемешивание электролита.
    \item увеличение продолжительности первого цикла анодного окисления позволяет увеличить размер доменов пористого оксида алюминия, однако превышение некоторого критического значения приводит к уменьшению среднего размера домена \cite{napolsky-sintez}, \cite{petukhov}.
\end{itemize}

\subsection{Математическая модель}

С точки зрения обоснования свойств полученных материалов, а также их прогнозирования, важным является математическое обоснование не только локальных напряжений (модель внутренних напряжений популярна для описания процессов формирования пористой структуры), но и проблема установления зависимости средних по материалу напряжений от значений локальных напряжений, обусловленных различиями термических коэффициентов линейного расширения (ТКЛР) между материалом нанокристалла и матрицы \cite{gasenkova}.

В работе \cite{gasenkova} принимают, что в рассматриваемом однонаправленно армированном композите компоненты изотропны, положение нитевидных нанокристаллов (волокон) в объёме матрицы является случайным, однако в целом материал предполагается статистически однородным. Из этого следует наличие среднего расстояния между волокнами, которое может быть связанно с их концентрацией.

Рассмотрим некоторый усредненный элементарный объем в виде правильной шестиугольной призмы, в центре которой находится один цилиндрический нитевидный нанокристалл, ориентированный вдоль оси $z$ системы координат.

Пусть отдельное нитевидное волокно имеет средний радиус $r$, а расстояние от центра правильного шестиугольника до его стороны равно $r+h$.

Тогда площадь основания элементарной ячейки будет равна:
\begin{equation}
S=2\sqrt{3}(r+h)^2    
\end{equation}
Площадь поперечного сечения волокна:
\begin{equation}
S_в=\pi r^2    
\end{equation}
Считая, что концентрация волокон равна:
\begin{equation}
v_\text{в}=S_\text{в}/S
\end{equation}
(здесь и далее индекс "$\text{в}$" обозначает величины, относящиеся к волокнам, а "$\text{м}$" - к матрице), получим:
\begin{equation}
v_\text{в}=\frac{\pi}{2\sqrt3(1+h/r)^2}  
\end{equation}
$\frac{h}{r}$ - параметр, характеризующий структуру композита, может быть выражен через концентрацию нитевидных кристаллов в виде:
\begin{equation}
\frac{h}{r}=\sqrt{\frac{\pi}{2\sqrt{3}v_\text{в}}-1}
\end{equation}
Максимально возможное значение концентрации нитевидных волокон будет в случае, когда $\frac{h}{r}\to0$, что соответствует выражению:
\begin{equation}
  v_\text{в}\to\frac{\pi}{2\sqrt{3}}\approx0.9  
\end{equation}
Минимальное значение концентрации волокон характеризует случай, когда $\frac{h}{r}\to\infty$, откуда $v_\text{в}\to0$. Данный диапазон концентрации нитевидных нанокристаллов соответствует границам применимости методов расчета свойств подобных материалов.

Используя данные расчеты можно прогнозировать возможность заполнения пор различными материалами, в том числе и $Ni$.

\subsection{Применение ПОА, синтезированного методом анодного оксидирования}

\subsubsection{Использование пористого $Al_2O_3$ для формирования наноструктур в порах матрицы}

\addimg{nanoniti}{1}{Микрофотографии (А) $Ni$ нанонитей с диаметром $60$ $\text{нм}$, (Б) $Pt$ нанотрубок (внешний диаметр $300$ $\text{нм}$), (В) нитевидных слоистых ($20$ $\text{нм}$ $Ni$, $10$ $\text{нм}$ $Cu$) наночастиц с диаметром $100$ $\text{нм}$ \cite{napolsky-sintez}.}{nanoniti}

Пористый оксид алюминия часто применяется в качестве матрицы для получения нитевидных наночастиц различного состава, слоистых нитевидных наночастиц, а также нанотрубок (рис. \ref{nanoniti}).

\subsubsection{Создание мембранных материалов на основе ПОА}

Другим перспективным классом материалов на основе пористого оксида алюминия являются мембраны для разделения газов и катализа. Мембраны на основе анодного оксида алюминия, получаемые методом анодного окисления, обладают уникальной микроструктурой, представленной плотной системой цилиндрических каналов, проходящих сквозь всю мембрану, с узким распределением по размерам и малой извилистостью. 

Важной особенностью также является возможность варьировать параметры структуры (расстояние между порами, диаметр пор, толщина мембраны), изменяя условия анодирования. Кроме того, существует возможность химической модификации поверхности стенок пор мембраны.  Одна из широко известных методик контроля диаметра пор и состояния поверхности их стенок состоит в покрытии стенок пор мембраны тонким слоем золота, на который адсорбируются различные тиолы. На поверхности стенок пор образуются самособирающиеся монослои, управление транспортом осуществляется за счёт варьирования состава функциональных групп самособирающихся монослоев \cite{petukhov}.

Иным вариантом модификации стенок пор мембраны является формирование в каналах пористой структуры анодного оксида алюминия углеродных нанотрубок. Следует отметить, что в дальнейшем поверхность синтезированных углеродных нанотрубок может быть модифицирована путем отжига в различных атмосферах (например, отжиг в аммиаке приводит к формированию на поверхности углеродных нанотрубок аминогрупп; отжиг в атмосфере влажного воздуха – формированию карбоксильных групп). Кроме того, необходимо отметить, что поверхность стенок пор в данном случае является проводящей, что позволяет прикладывать потенциал к мембране, что также может быть использовано для разделения различных биомолекул \cite{petukhov}. Одним из примеров таких мембран являются нанокомпозиты на основе пористого оксида алюминия и соединений серебра, которые могут использоваться для обеззараживания воды \cite{zolotukhina-voda}, \cite{zolotukhina-nanoporous}.

Следующим интересным примером служат композиты на основе ПОА, модифицированного солями некоторых $d$-металлов и углеродными нанотрубками, которые могут найти применение в качестве сорбента аммиака \cite{vorobyova-ammiak}. 

В качестве завершающего примера на рис. \ref{pdpoa} представлены нанокомпозиты на основе палладия $Pd/Al_2O_3$, которые могут использоваться для очистки водорода.

\addimg{pdpoa}{1}{Микрофотография (СЭМ) нанокомпозита $Pd/Al_2O_3$ ($d_\text{пор}=60\text{нм}$) после растворения матрицы \cite{napolsky-sintez}}{pdpoa}

\subsubsection{Гетероструктуры на основе наностолбиков Ni в ПОА}

Такие металлы, как $Pd$, $Pt$ и $Ni$ нашли широкое применение в области катализа. Мембраны на их основе используются уже давно, однако, поиск новых применений происходит и по сей день \cite{guaiacol}. В работе \cite{buhtiyarov} подробно исследуются каталитические свойства мембран на основе $Ni$.

Одно из применений подобных мембран - каталитический риформинг, как метод удаления смолы. В этом контексте он является экологически благоприятным и высокоэффективным методом, так как почти не образует отходов. Так же, катализаторы на основе переходных металлов, особенно катализаторы на основе $Ni$, обратили на себя пристальное внимание в качестве катализаторов, используемых при газификации \cite{buhtiyarov}, \cite{guaiacol} так как они являются возобновляемыми.

Для создания никелевых катализаторов частичы $Ni$ обычно наносят на пористые носители, такие как оксид алюминия. $Ni/Al_2O_3$ очень активен в паровой конверсии небольших органических веществ, таких как этанол, уксусная кислота, ацетон и т.д. \cite{buhtiyarov}. На основе данной работы можно сделать вывод о том, что никель является перспективным материалом для катализа различных органических реакций. Несомненным преимуществом дальнейшего исследования нанокомпозитов никеля является наличие достаточной фундаментальной базы, в сравнении с которой удобно проводить анализ процессов и механизмов синтеза.

Еще одним аргументом в пользу выбора никеля для исследования являются его магнитные свойства. Наностолбики из ферромагнитных материалов ($Ni$, $Co$, $Fe$), инкорпорированные в матрицу из диэлектрического материала, являются предметом интенсивных исследований последнего десятилетия. Интерес к подобным объектам вызван тем, что на них наблюдается эффект колоссального магнетосопротивления, когерентных спиновых волн и т.д. Отличительной особенностью таких нанообъектов является их высокая анизотропия. Все это в совокупности делает их весьма перспективными для создания устройств магнитной памяти с высокой плотностью упаковки\cite{ziganshina}.

Другая сфера применеия - оптика. В даный момент микроскопические элементы вызывают большой интерес в области оптических телекоммуникаций и оптических сенсоров. Микрополяризатор - один из элементов, необходимый в таких оптических устройствах, как изоляторы и переключатели. Авторы работы \cite{saito} предложили новый тип микрополяризатора на основе ПОА, который содержит множество цилиндрических макропор, ориентированных параллельно друг к другу. Ожидается, что имея металлические проводники, такие как $Ni$, в порах, гетероструктура будет работать как поляризатор.

Еще одно преимущество устройств подобного рода - относительная простота их строения.
ПОА с микропорами столбчатой структуры является весьма привлекательным материалом для использования в оптических приборах \cite{kolmychek}, \cite{kolmychek2}. На сегодняшний день плёнки пористого оксида алюминия исследованы в качестве источников для создания композитных материалов, с помощью которых были осуществлены такие оптические функции, как спектральная селекция, поляризация и оптическое переключение, с помощью заполнения пор такими материалами, как металлы и полимеры. Кроме того, было установленно, что ПОА демонстрирует поляризационную функцию сама по себе, без участия заполняющих поры материалов. Поляризационная функия объясняется предположением о наличии частиц не оксидированного $Al$ путём химического анализа плёнок ПОА.

\subsection{Выводы}

Изучена научно-техническая литература по заявленной теме, выявленны основные подходы к созданию нанокомпозитов $Ni$/ПОА, найдены основные формулы для расчета диаметра пор, количества нанонитей в порах и их размеров. 
Основываясь на экспериментальных и рассчетных данных из работ, представленных в литературном обзоре, можно сделать предположение об оптимальных методиках синтеза гетероструктур $Ni$/ПОА.

\begin{itemize}
    \item Исходя из представленных изображений пор с иерархической структурой, можно сделать вывод о том, что предпочтительнее использовать режим анодирования с постоянным значением напряжения тока в ходе процесса, так как при уменьшении значения напряжения в $\sqrt{n}$ раз происходит ветвление одной поры на $n$ пор;
    \item Применяя формулы из работы \cite{gasenkova}, можно прогнозировать возможность заполнения объема пор различными материалами, в том числе и $Ni$, рассчитать степень заполнения пор нитевидными волокнами $Ni$;
    \item Оптимальный режим для создания упорядоченных пор: $U=40\text{В}$; $j=0,65 \text{А/см}^2$;
    \item При малых диаметрах пор возрастает величина адгезии $Ni$ к стенкам, происходит дальнейший рост к центру поры и, как следствие, при дальнейшем увеличении слоя $Ni$ образуется тонкая плёнка, повторяющая поверхностную структуру темплата;
    \item Росту упорядоченных нанонитей в порах темплата может способствовать предварительная подготовка ПОА, а именно - выдерживание матриц при температуре $400^\circ C$. Этот процесс способствует устранению пузырьков газа и паров воды из полостей пор;
    \item Для получения отдельных наностержней $Ni$ небольшого диаметра ($\sim$ диаметра пор), необходимо использовать электролит с высокой скоростью травления ($1M$ $NaOH$);
    \item{Гексагональная геометрия, характерная для пор оксида алюминия, не сохраняется при толщине слоя $Ni$ на поверхности ПОА большей, чем $2 \text{мкм}$.}
\end{itemize}

\clearpage
