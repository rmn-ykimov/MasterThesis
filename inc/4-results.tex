\section{Анализ полученных результатов}

В результате работы были получены пленки пористого $Al_2O_3$ с $Ni$ в объёме пор. Исходным материалом для синтеза ПОА были алюминиевые фольги $A7$ (ГОСТ 11060-2001) толщиной $30\text{мкм}$, предварительно отожжённые в трубчатой печи для рекристаллизации и снятия локальных микронапряжений. Предварительный отжиг позволил уменьшить количество примесей в зерне (примеси скапливаются у границ зерен). Методом двухстадийного двустороннего анодирования при комнатной температуре был получен темплат, который затем был разделен на 2 части. Первая часть (образец 1) была использована для исследования структуры исходного ПОА методами АСМ. Вторая половина была использована для запыления $Ni$ в объем пор методом магнетронного напыления. Вторая часть темплата, на которую методом магнетронного напыления был осажден $Ni$, также была разделена на 2 половины, одна из которых (образец 2) была использована для исследования поверхностной морфологии методами АСМ, а вторая часть (образец 3) - для исследования скола с помощью РЭМ. Для выбора участков съемки были использованы изображения оптического микроскопа.

Для анализа изображений были использованы следующие сигналы АСМ (контактный режим):

\begin{itemize}
    \item DFL - режим постоянной силы
    \item LF - латеральные деформации кантилевера
    \item Mag - величина амплитуды нормальных колебаний
    \item Phase - распределение фазы
\end{itemize}

\subsection{Результаты исследований, полученные методом оптической микроскопии}

\addimg{cantilever}{0.5}{Оптическое изображение кантилевера. L = $100 \pm{5} \text{мкм}$, W = $35 \pm{5} \text{мкм}$ }{cantilever}

\addimg{optics}{1}{а) Оптическое изображение участка ПОА до запыления слоя $Ni$ толщиной $50 \text{нм}$. б) Оптическое изображение участка ПОА после запыления слоя $Ni$ толщиной $50 \text{нм}$}{optics}

Изображения \ref{optics} а., \ref{optics} б. получены с помощью оптического микроскопа. Они позволяют судить о неоднородности поверхности исследуемого образца. Продольные борозды на рис. \ref{optics} а. и рис. \ref{optics} б. обусловлены тем, что в качестве исходного материала для синтеза ПОА была выбрана катанная фольга, они расположенны параллельно направлению проката. Также заметны последствия травления образцов в $H_3PO_4$ (этап электрохимической полировки). Травление происходило с неодинаковой скоростью из-за различной ориентации зёрен $Al$ рис. \ref{optics} а. По этой же причинине происходит углубление борозд \ref{optics} б. Пятна, которые можно наблюдать на рис. \ref{optics} б., являются следствием недостаточнойй промывки образцов после стравливания $Al$. Выбор места дальнейшего анализа методами АСМ основывался на изображениях оптического микроскопа. Кантилевер представлен на рис.\ref{cantilever}. Его длина ($L$) равна $100 \pm{5} \text{мкм}$, ширина ($W$) - $35 \pm{5} \text{мкм}$. Изображения \ref{optics} а., \ref{optics} б. не имеют значительных визуальных отличий (текстура поверхности не изменилась). Это может быть связано с тем, что интегральная толщина слоя $Ni$ ($50 \text{нм}$) не отражается на оптических свойствах исследуемой поверхности, так как спектр видимого света значительно превышает толщину слоя $Ni$.

\subsection{Результаты исследований, полученные методом атомно-силовой микроскопии (до напыления $Ni$)}

\addimg{10x10_before}{1}{а) АСМ-изображение рельефа поверхности ПОА, синтезированного двусторонним двухстадийным анодированием в потенциостатическом режиме при U=40В. Размер области сканирования 10х10 мкм. Образец 1, участок 1. б) Гистограмма распределения высот для образца 1, участка 1}{10x10_before}

\addimg{5x5_before}{1}{а) АСМ-изображение рельефа поверхности ПОА, увеличенная область участка 1. Размер области сканирования 5х5 мкм. Образец 1, участок 2. б) АСМ-изображение рельефа поверхности ПОА, DFL режим. Размер области сканирования 5х5 мкм. Образец 1, участок 2. в) Гистограмма распределения высот для образца 1, участка 2}{5x5_before}

\addimg{2.5x2.5_before}{1}{а) АСМ-изображение рельефа поверхности ПОА, увеличенная область участка 2. Размер области сканирования 2.5х2.5 мкм. Образец 1, участок 3. б) АСМ-изображение рельефа поверхности ПОА, LF режим. Размер области сканирования 2.5х2.5 мкм. Образец 1, участок 3. в) Гистограмма распределения высот для образца 1, участка 3}{2.5x2.5_before}

\addimg{1x1_before}{1}{а) АСМ-изображение рельефа поверхности ПОА, увеличенная область участка 3. Размер области сканирования 1х1 мкм. Образец 1, участок 4. б) АСМ-изображение рельефа поверхности ПОА, LF режим. Размер области сканирования 1х1 мкм. Образец 1, участок 4. в) Гистограмма распределения высот для образца 1, участка 4}{1x1_before}

\addimg{2x2_before}{1}{а) АСМ-изображение рельефа поверхности ПОА, синтезированного двусторонним двухстадийным анодированием в потенциостатическом режиме при U=40В. Размер области сканирования 2х2 мкм. Образец 1, участок 5. б) АСМ-изображение рельефа поверхности ПОА, DFL режим. Размер области сканирования 2х2мкм. Образец 1, участок 5. в) Гистограмма распределения высот для образца 1, участка 5. г) Высотный профиль отрезка на участке 5, образец 1}{2x2_before}

\clearpage

На основании данных снимков \ref{10x10_before} - \ref{2x2_before} была составленна таблица:

\begin{table}[H]
    \caption{Параметры рельефа ПОА до напыления $Ni$}\label{table_before}
    \begin{tabular}{|l|l|l|l|}
    \hline Рисунок, № & Площадь, $\text{мкм}^2$ & Перепад высот (ср), нм & Шероховатость  (ср), нм \\
    \hline \ref{10x10_before}, а & 100.06 & 135.68 & 9.69 \\
    \hline \ref{5x5_before}, а & 24.11 & 92.29 & 8.69 \\
    \hline \ref{2.5x2.5_before}, а & 5.59 & 105.35 & 9.14 \\
    \hline \ref{2x2_before}, а & 4.01 & 78.04 & 8.45 \\
    \hline \ref{1x1_before}, а & 0.83 & 71.22 & 8.41 \\
    \hline
    \end{tabular}
\end{table}

Изображения \ref{5x5_before}, б., \ref{2x2_before}, б. получены с помощью метода DFL (режим постоянной силы). Значение DFL прямо пропорционально силе взаимодействия кантилевера с поверхностью образца. Оно тем больше, чем больше вертикальное смещение кантилевера. Этот метод позволяет различить, стенки и внутренню часть пор (чем темнее цвет области на изображении, тем ниже был опущен кантилевер во время сканированя).

Изображения \ref{2.5x2.5_before}, б, \ref{1x1_before}, б, полученные в режиме LF (режим латеральных деформаций кантилевера) позволяют сделать вывод о том, изгиб кантилевера увеличивается на границах поры, что отражается на силе сигнала (чем светлее область на изображениях, тем сильнее латеральные деформации). Из этого можно сделать вывод о том, что фрикционные свойства стенок пор отличаются от тех же свойств самой поры.

На основании рис. \ref{2x2_before}, г. был рассчитан тангенс угла наклона касательной к стенке поры. Его значение равно 4 (угол равен $76^{\circ}$ ).

Таким образом, на основании снимков, приведённых выше, можно сделать вывод о том, что поры имеют гексагональную форму. Два метода (DFL и LF) независимо друг от друга подтвердили один и тот же результат: образец обладает пористой структурой. Помимо этого, на основе данных таблицы \ref{table_before} были рассчитанны средние значения для следующих параметров исходного ПОА:

\begin{itemize}
    \item перепад высот (ср) ($\sim 96,51\text{нм}$)
    \item диаметр пор (ср) ($\sim 80\text{нм}$)
    \item шероховатость (ср) ($\sim 8,87\text{нм}$)
\end{itemize}

\subsection{Результаты исследований, полученные методом атомно-силовой микроскопии (после напыления $Ni$)}

\addimg{2x2_after}{1}{а) АСМ-изображение рельефа поверхности ПОА c Ni, осажденным методом магнетронного напыления (толщина слоя 50 нм). Размер области сканирования 2х2 мкм. Образец 2, участок 1. б) АСМ-изображение рельефа поверхности ПОА после применения FTT filtration. Размер области сканирования 2х2мкм. Образец 2, участок 1. в) АСМ-изображение рельефа поверхности ПОА, Mag режим. Размер области сканирования 2х2мкм. Образец 2, участок 1. г) Гистограмма распределения высот для образца 2, участка 1}{2x2_after}

\addimg{height_profile_2x2_after}{1}{а) АСМ-изображение рельефа поверхности ПОА c Ni, осажденным методом магнетронного напыления (толщина слоя 50 нм). Размер области сканирования 2х2 мкм. Образец 2, участок 1. б) Высотный профиль отрезка на участке 1, образец 2}{height_profile_2x2_after}

\addimg{1.2x1.2_after}{1}{а) АСМ-изображение рельефа поверхности ПОА, увеличенная область участка 1. Размер области сканирования 1х1 мкм. Образец 2, участок 2. б) АСМ-изображение рельефа ПОА, Phase режим. Размер области сканирования 1х1мкм. Образец 2, участок 2. в) Гистограмма распределения высот для образца 2, участка 2}{1.2x1.2_after}

\addimg{20x20_after}{1}{а) АСМ-изображение рельефа поверхности ПОА. Размер области сканирования 20х20 мкм. Образец 2, участок 3. б) АСМ-изображение рельефа поверхности ПОА, Phase режим. Размер области сканирования 20х20мкм. Образец 2, участок 3.}{20x20_after}

\addimg{5x5_after}{1}{а) АСМ-изображение рельефа поверхности ПОА. Размер области сканирования 5х5 мкм. Образец 2, участок 4. б) АСМ-изображение рельефа поверхности ПОА, Phase режим. Размер области сканирования 5х5мкм. Образец 2, участок 4. в) Гистограмма распределения высот для образца 2, участка 4}{5x5_after}

\clearpage

На основании данных снимков \ref{2x2_after} - \ref{5x5_after} была составленна таблица:

\begin{table}[H]
    \caption{Параметры рельефа ПОА после напыления $Ni$}\label{table_after}
    \begin{tabular}{|l|l|l|l|}
    \hline Рисунок, № & Площадь, $\text{мкм}^2$ & Перепад высот (ср), нм & Шероховатость  (ср), нм \\
    \hline \ref{20x20_after}, а & 407.52 & 107.01 & 6.49 \\
    \hline \ref{5x5_after}, а & 22.36 & 78.49 & 7.04 \\
    \hline \ref{2x2_after}, а & 4.01 & 108.14 & 9.91 \\
    \hline \ref{1.2x1.2_after}, а & 1.19 & 73.91 & 8.71 \\
    \hline
    \end{tabular}
\end{table}

На рисунках (\ref{2x2_after}-\ref{5x5_after}) представлены АСМ изображения поверхности того же образца ПОА после магнетронного напыления $Ni$. Для анализа изображений было использованно несколько режимов АСМ, а именно:

\begin{itemize}
    \item Mag - величина амплитуды нормальных колебаний
    \item Phase - распределение нормальной фазы
\end{itemize}

На рис. \ref{2x2_after}, б. - результат фурье-преобразования для рис. \ref{2x2_after}, а. Это позволило различить структуру изображения и избавиться от экстремумов, появлявшихся на исходном изображении. До применения фильтра максимальная высота рельефа была равна $108.6 \text{нм}$, после - $72.6 \text{нм}$. На основании изображения \ref{2x2_after}, в., полученном в режиме Mag, можно заключить, что  амплитуда нормальных колебаний увеличивается к центру пор и уменьшается на границах.

Изображения \ref{1.2x1.2_after}, \ref{20x20_after}, \ref{5x5_after} получены с применением режима Phase. Фазовый сдвиг колебаний зависит от материала поверхности. Чем он больше, тем больше силы взаимодействия образца с кантилевером. Более светлые области изображения соответствуют тем областям поверхности ПОА, где силы взаимодействия больше (границы пор). Это свидетельствует об изменении фазы.

На основании рис. \ref{height_profile_2x2_after}, б. был рассчитан тангенс угла наклона касательной к стенке поры. Его значение равно 0.58 (угол равен $30.5^{\circ}$).

Основываясь на снимках, приведённых выше, можно сделать вывод о том, что  $Ni$ был осаждён равномерно на всю поверхность образца, включая стенки пор. Ввиду малой толщины осажденного слоя, наблюдается не сплошное зарастание пор. На основе данных таблицы \ref{table_after} были рассчитанны средние значения для следующих параметров исходного ПОА:

\begin{itemize}
    \item перепад высот (ср) ($\sim 91,88 \text{нм}$)
    \item диаметр пор (ср) ($\sim 65\text{нм}$)
    \item шероховатость (ср) ($\sim 8,03\text{нм}$)
\end{itemize} 

\subsection{Результаты исследований, полученные методом РЭМ (после напыления $Ni$)}

\addimg{РЭМ ПОА + Ni}{0.6}{РЭМ - изображение скола ПОА. Увеличение - 7500}{РЭМ ПОА + Ni}

Основываясь на изображении скола ПОА, полученного с помощью РЭМ (рис.\ref{РЭМ ПОА + Ni}) можно приблизительно определить внутренний диаметр пор. Он составляет от $50\text{нм}$ до $60\text{нм}$. Также можно утверждать, что поры не имеют иерархической структуры, не имеют ветвлений, их диаметр не изменяется по всей длине. Толщина стенок поры составляет от $10\text{нм}$ до $30\text{нм}$.
Косвенно о наличии $Ni$ в объеме пор можно судить, учитывая, что светлые поля изображения создаются металлом. Они возникают из-за вторичных электронов, испущенных $Ni$, темные поля - оксидом. Светлые поля распределяются вдоль поверхности ПОА и в объме пор, из чего можно предположить, что $Ni$ равномерно заполнил их, часть была осаждена на стенки пор (светлые поля в поверхностном слое).

\subsection{Выводы}

Съемка в различных режимах АСМ (Contact, DFL, LF) показала, что образец имеет пористую структуру. Об этом же говорит РЭМ изображение. Диаметр пор, рассчитанный на основе снимков АСМ совпадает с диаметром, рассчитанным на основе снимка РЭМ и составляет $\sim 80 \text{нм}$.

Приведённые выше данные позволяют говорить о том, что $Ni$ заполнил объем пор и был осажден на поверхность ПОА. Данный вывод можно сделать, исходя из следующих данных: при нанесении на ПОА $Ni$ среднее значение перепада высот изменилось незначительно, от $96,51 \text{нм}$ до $91,88 \text{нм}$, среднее значение диаметра пор изменилось от $\sim 80\text{нм}$ до $\sim 65\text{нм}$. Среднее значение шероховатости изменилось от $\sim 8,87\text{нм}$ до $\sim 8,03\text{нм}$. Кроме того, о наличии $Ni$ в объеме пор можно судить, опираясь на данные РЭМ (распределение светлых полей вдоль поверхности ПОА и в объме пор). Также, косвенным свидетельством наличия $Ni$ в объёме пор является изменение тангенса угла наклона касательной к стенке пор, до запыления значение было равно 4 (угол равен $76^{\circ}$) после запыления - 0.58 (угол равен $30.5^{\circ}$). Его уменьшение связанно с нарастанием $Ni$ на стенках пор.

\clearpage
