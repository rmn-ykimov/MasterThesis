\begingroup
\renewcommand{\section}[2]{\anonsection{Библиографический список}}
\begin{thebibliography}{00}

\bibitem{belov-osobennosti}
Белов А.Н. Особенности получения наноструктурированного анодного оксида алюминия / А.Н. Белов, С.А. Гаврилов, В.И. Шевяков // Российские нанотехнологии. – 2006. – Т. 1, вып. №1-2. – С.223-225.

\bibitem{atraschenko-hyperbolyc}
Атращенко А. В. и др. Электрохимические методы синтеза гиперболических метаматериалов //Наносистемы: физика, химия, математика. – 2012. – Т. 3. – №. 3. – С. 33-54.

\bibitem{yakovleva-zakonomernosti}
Яковлева Н.М. Структурно-морфологические закономерности формирования нанопористых оксидов алюминия: Автореф. дис. на соиск. учен. степ. д.ф.-м.н.: Спец. 01.04.07 / Яковлева Наталья Михайловна; [Воронеж. гос. техн. ун-т]. – Воронеж, 2003. - 31 с

\bibitem{viharev-osobennosti}
Вихарев А. В., Вихарев А. А. Особенности строения и механизм формирования анодных оксидов алюминия //Ползуновский вестник. – 2010. – №. 3. – С. 204-208.

\bibitem{li_y}
Li Y. et al. Effect of hydrothermal treatment on porous anodic alumina generated by one-step anodization //Microporous and Mesoporous Materials. – 2020. – Т. 306. – С. 110412.

\bibitem{grilihes-pokrytiya}
Грилихес С. Я. Оксидные и фосфатные покрытия металлов/Под. ред. ПМ Вячеславова.–5-е изд., перераб. и доп //Л.: Машиностроение, Ленинградское отд-е. – 1985.

\bibitem{napolskiy-avtoref}
Напольский К. С. Электрохимическое формирование пространственно-упорядоченных металлических наноструктур в пористых матрицах //МГК им. МВ Ломоносова: Москва. – 2009.

\bibitem{varipaev-prakticum}
Варыпаев В.Н. Практикум по прикладной электрохимии / В.Н. Варыпаев, В.Н. Кудрявцев (ред.). –Учебное пособие для вузов. 3-е издание, перераб. - Л.: Химия, 1990. - 304 с.

\bibitem{yang-influence}
Yang Y., Gao Q. Influence of sulfosalicylic acid in the electrolyte on the optical properties of porous anodic alumina membranes //Physics letters A. – 2004. – Т. 333. – №. 3-4. – С. 328-333.

\bibitem{mombello}
Mombello D. et al. Porous anodic alumina for the adsorption of volatile organic compounds //Sensors and actuators B: chemical. – 2009. – Т. 137. – №. 1. – С. 76-82.

\bibitem{wang-analysis}
Wang H., Yi H., Wang H. Analysis and self-lubricating treatment of porous anodic alumina film formed in a compound solution //Applied surface science. – 2005. – Т. 252. – №. 5. – С. 1662-1667.

\bibitem{coz-analysis}
Le Coz F., Arurault L., Datas L. Chemical analysis of a single basic cell of porous anodic aluminium oxide templates //Materials characterization. – 2010. – Т. 61. – №. 3. – С. 283-288.

\bibitem{grigoryev-2d}
Григорьев С. В. и др. Двумерные пространственно-упорядоченные системы $Al _2 O _3$: исследование методом малоуглового рассеяния нейтронов //Письма в Журнал экспериментальной и теоретической физики. – 2007. – Т. 85. – №. 9. – С. 549-554.

\bibitem{jaafar-alumina}
Jaafar M. et al. Nanoporous alumina membrane prepared by nanoindentation and anodic oxidation //Surface science. – 2009. – Т. 603. – №. 20. – С. 3155-3159.

\bibitem{napolsky-sintez}
Напольский К. С. 3. СИНТЕЗ ПРОСТРАНСТВЕННО УПОРЯДОЧЕННЫХ МЕТАЛЛ-ОКСИДНЫХ НАНОКОМПОЗИТОВ НА ОСНОВЕ ПОРИСТОГО Al2O3 //Химические методы синтеза неорганических веществ и материалов. – 2008. – С. 69.

\bibitem{petukhov}
Петухов Д. И. Создание мембранных материалов на основе анодного оксида алюминия //Москва: МГУ им. МВ Ломоносова. – 2011.

\bibitem{evertsson}
Evertsson J. et al. Self-organization of porous anodic alumina films studied in situ by grazing-incidence transmission small-angle X-ray scattering //RSC advances. – 2018. – Т. 8. – №. 34. – С. 18980-18991.

\bibitem{gasenkova}
Гасенкова И. В., Андрухович И. М. Использование анодного оксида алюминия при электрохимическом осаждении никеля //Современные электрохимические технологии и оборудование-2019. – 2019. – С. 361-365.

\bibitem{buhtiyarov}
Бухтияров В. И. и др. Европейский конгресс по катализу //Каталитический бюллетень. – 2015. – Т. 76. – С. 1-61.

\bibitem{guaiacol}
Zhang Z. et al. Steam reforming of guaiacol over Ni/Al2O3 and Ni/SBA-15: Impacts of support on catalytic behaviors of nickel and properties of coke //Fuel Processing Technology. – 2019. – Т. 191. – С. 138-151.

\bibitem{ziganshina}
Зиганшина С. А. и др. Наночастицы никеля, сформированные в трековых порах, для электрокатализа //Казанский физико-технический институт имени Е. К Завойского. Ежегодник. – 2016. – Т. 2015. – С. 45-48.

\bibitem{saito}
Saito M. et al. Micropolarizer made of the anodized alumina film //Applied physics letters. – 1989. – Т. 55. – №. 7. – С. 607-609.

\bibitem{kolmychek}
Kolmychek I. A. et al. Magneto-optical effects in hyperbolic metamaterials //Optics letters. – 2018. – Т. 43. – №. 16. – С. 3917-3920.

\bibitem{kolmychek2}
Kolmychek I. A. et al. Magneto-Optical Effects in Au/Ni Based Composite Hyperbolic Metamaterials //Physics of Metals and Metallography. – 2019. – Т. 120. – №. 13. – С. 1266-1269.

\bibitem{masuda-mosaic}
Masuda H. et al. Ordered mosaic nanocomposites in anodic porous alumina //Advanced materials. – 2003. – Т. 15. – №. 2. – С. 161-164.

\bibitem{napolsky-samoorganizatsiya}
Напольский К.С. In-situ изучение процесса самоорганизации пористой структуры анодного оксида алюминия. – МГУ им. Ломоносова кафедра неорганической химии, Петербургский институт ядерной физики им. Б.П. Константинова РАН, г. Гатчина, Debye Institute for Nanomaterials, University of Utrecht, The Netherland, 2009.

\bibitem{sheasby-treatment}
Sheasby P. G., Wernick S., Pinner R. Surface treatment and finishing of aluminum and its alloys. Volumes 1 and 2. – 1987.

\bibitem{patermarakis-investigation}
Patermarakis G., Moussoutzanis K., Nikolopoulos N. Investigation of the incorporation of electrolyte anions in porous anodic Al2O3 films by employing a suitable probe catalytic reaction //Journal of solid state electrochemistry. – 1999. – Т. 3. – №. 4. – С. 193-204.

\bibitem{wehrspohn-10rule}
Nielsch K. et al. Self-ordering regimes of porous alumina: the 10 porosity rule //Nano letters. – 2002. – Т. 2. – №. 7. – С. 677-680.

\bibitem{suminov-modifitsirovanie}
Суминов И. и др. (ред.). Плазменно-электролитическое модифицирование поверхности металлов и сплавов. В 2 томах. Том 2. – Litres, 2022. – Т. 2.

\bibitem{sacco}
Sacco L., Florea I., Cojocaru C. S. Fabrication of porous anodic alumina (PAA) templates with straight pores and with hierarchical structures through exponential voltage decrease technique //Surface and Coatings Technology. – 2019. – Т. 364. – С. 248-255.

\bibitem{solovey-polevie}
Соловей Д. В., Горох Г. Г., Сахарук В. Н. Полевые эмиссионные катоды на основе пористого анодного оксида алюминия и углеродных нанотрубок //Наносистемы, наноматериалы, нанотехнологии. – 2011

\bibitem{rakov-materiali}
Раков Э. Г. Материалы из углеродных нанотрубок." Лес" //Успехи химии. – 2013. – Т. 82. – №. 6. – С. 538-566.

\bibitem{rakov-fullereni}
Раков Э. Г. Нанотрубки и фуллерены: учеб. пособие для студ., обуч. по спец. 210602" Наноматериалы"/ЭГ Раков.-Москва //Логос. – 2006. – Т. 21. – С. 359-369.

\bibitem{meng-nanowires}
Meng G., Ajayan P. M., Jung Y. J. Controlled fabrication of hierarchically branched nanopores, nanotubes, and nanowires : заяв. пат. 11435294 США. – 2010.

\bibitem{zolotukhina-voda}
Золотухина Е. В. и др. Обеззараживание воды нанокомпозитами на основе пористого оксида алюминия и соединений серебра //Сорбционные и хроматографические процессы. – 2010. – Т. 10. – №. 1. – С. 78-85.

\bibitem{zolotukhina-nanoporous}
Золотухина Е. В. и др. Электрохимическое поведение нанопористого анодированного алюминия в воде с микроорганизмами //Вестник Воронежского государственного технического университета. – 2010. – Т. 6. – №. 10. – С. 7-11.

\bibitem{vorobyova-ammiak}
Воробьев А. Ю. и др. Адсорбция аммиака композитами на основе нанопористого оксида алюминия //Вестник Воронежского государственного технического университета. – 2012. – Т. 8. – №. 7-2. – С. 4-7.

\bibitem{vacuumnoe}
Вдовичев С. Н. Современные методы высоковакуумного напыления и плазменной обработки тонкопленочных металлических структур //Нижний Новгород: Нижнегородский госуниверситет. – 2012.

\bibitem{kelcall}
Келсалл Р. Научные основы нанотехнологий и новые приборы. Учебник-монография. Пер. с англ.: Научное издание / Р. Келсалл, А. Хэмли, М. Геогеган. – Долгопрудный: Издательский Дом «Интеллект», 2011. – 528 с.

\bibitem{eliseev-functsionalnie}
 Елисеев А.А. Функциональные наноматериалы. Под редакцией Ю.Д. Третьякова / А.А. Елисеев, А.В. Лукашин. – М.: ФИЗМАТЛИТ, 2010. – 456 с.

\end{thebibliography}
\endgroup

\clearpage
