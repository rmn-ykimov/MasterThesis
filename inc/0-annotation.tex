\annotation{Аннотация}

Тема выпускной квалификационной работы магистра --- <<Синтез гетероструктуры $Ni$/ПОА методом магнетронного распыления>>.

Ключевые слова: ПОА, $Ni$, темплат, магнетронное распыление, гетероструктура.

Актуальность темы.

Технология анодирования начала активно применяться уже давно, однако в последние годы она стала интенсивно развиваться в связи с возможностью применения этого процесса для получения пористых плёнок с заданной морфологией и, как следствие, с заранее определёнными характеристиками. Пористые анодные оксидные плёнки могут быть синтезированы на основе различных материалов, таких как кремний, фосфид индия, титан, ниобий, тантал, олово и т.д.

На сегодняшний день плёнки пористого оксида алюминия используются главным образом как матрицы для синтеза упорядоченных массивов анизотропных наноструктур различного состава. Внимание к таким структурам обращено в связи с возможностью как изучения на их примере фундаментальных задач (процессов самоорганизации и магнетизма в пространственно-упорядоченных наносистемах), так и решения широкого спектра прикладных проблем, касающихся создания высокоэффективных гетерогенных катализаторов, гиперболических метаматериалов, а также получения магнитных нанокомпозитов для устройств хранения информации со сверхвысокой плотностью записи.

Цель проводимых исследований - выявление оптимального алгоритма синтеза гетероструктур $Ni$/ПОА.
Для достижения намеченной цели необходимо разработать воспроизводимые методики синтеза однородного ПОА с заданными параметрами, провести запыление $Ni$ и экспериментальные исследования.

Выпускная квалификационная работа магистра изложена на \pageref{LastPage} листах, включает 2 таблицы, 29 рисунков, 41 литературный источник.

\clearpage
