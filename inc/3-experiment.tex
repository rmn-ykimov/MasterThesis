\section{Экспериментальные исследования}

В разделе экспериментальных исследований описаны стадии, необходимые для синтеза ПОА. Подраздел методы исследования посвящен описанию методов микроскопического анализа, применявшихся в работе.

\subsection{Методика терморезистивного отжига алюминиевых фольг}

Терморезистивный отжиг катанной алюминиевой фольги толщиной $30 \text{мкм}$ производился в камере трубчатой печи втоке азотав токе азота, во избежание образования на поверхности соединений с кислородом и углеродом из воздуха c целью снижения дисперсности зёренной структуры алюминиевой фольги, то есть увеличения размера кристаллитов алюминия и в последующем достижения лучшей упорядоченности пор оксида алюминия. К преимуществам данного метода можно отнести его простоту и дешевизну в виду отсутствия необходимости создавать глубокий вакуум. Для создания максимального количества областей, имеющих приблизительно одинаковую температуру, было принято решение отжигать образцы фольги длиной $30\text{см}$ и шириной $2,5\text{см}$. Выбор ширины был обусловлен размерами камеры трубчатой печи. После получения ленты алюминиевой фольги нужного размера, производилось промывание в ацетоне для удаления жировых пятен с поверхности. Экспериментальным путём был установлен оптимальный режим отжига: $4$ часа при нагреве до $600^{\circ} C$.


\subsection{Методика электрохимической полировки образцов}

Электрохимическая полировка предварительно отожжённых алюминиевых фольг осуществлялась в электрохимической ячейке. Катод и анод были подключены к источнику постоянного тока Agilent N8740A (рис.\ref{current_source}). Для поддержания условий, описанных в методике, была использована магнитная мешалка с регулированием температуры.

Перед началом работы электрический нагреватель выводился на рабочий режим ($80^{\circ}С$, $600$  $\text{об./мин.}$). Предварительно отожжённая фольга помещалась между двумя алюминиевыми катодами, на одинаковое расстояние от каждого из них. Полировка осуществлялась в импульсном режиме. Было подано $40$ импульсов по $3$ секунды, интервал между импульсами составлял $40$ секунд. По окончании полировки образец обильно промывался дистиллированной водой, сушился в эксикаторе. После полного удаления следов влаги, образец подвергался анодированию.

\subsection{Методика двустороннего анодирования при комнатной температуре}

Анодирование проводилось в $2$ стадии со снятием "жертвенного слоя" между ними. После окончания второй стадии осуществлялось открытие пор и стравливание оставшегося алюминия.

\subsubsection{Первая стадия}

Анодное окисление алюминия проводилось в двухэлектродной электрохимической ячейке с использованием источника постоянного тока Agilent N8740A (рис.\ref{current_source}) при постоянном перемешивании. Для регистрации изменения силы тока в процессе анодирования использовался цифровой мультиметр Agilent 34401a (рис.\ref{multimeter}).

\addimg{current_source}{1}{Источник постоянного тока Agilent N8740A}{current_source}

\addimg{multimeter}{0.6}{Цифровой мультиметр Agilent 34401а}{multimeter}

В данной работе в качестве электролита использовалась щавелевая кислота. Анодирование проводилось в потенциостатическом режиме, $U=40\text{В}$; $j=0,65 \text{А/см}^2$; $S_\text{образца} = 8\text{см}^2$. Длительность первого анодирования составляла $15$ минут, так как за это время образовывался барьерный слой оксида («жертвенный слой»), о чём свидетельствовало резкое падение силы тока, который препятствовал дальнейшему образованию пористой структуры.

\subsubsection{Снятие «жертвенного» слоя}

На начальной стадии процесса образующиеся поры малоупорядочены (рис.\ref{two_staged} в). Однако, в результате действия сил отталкивания между соседними порами, в ходе первого длительного окисления происходит самоупорядочение пористой структуры. Как итог -  на границе раздела оксид/металл образуется периодическая структура с плотнейшей гексагональной упаковкой пор в $Al_2O_3$ (рис.\ref{two_staged} д). После первого анодного окисления плёнку $Al_2O_3$ растворяют в смеси $H_3PO_4$, нагретой до $80^{\circ}С$, не затрагивая слоя $Al$, чтобы получить реплику нижней части оксидной плёнки, имеющей упорядоченную структуру.

\subsubsection{Второе анодирование}

Длительность второго анодирования составляла $1$ час. Электролит не менялся. Полученная плёнка обильно промывалась дистиллированной водой, а после – ацетоном.

\subsubsection{Открытие пор}

Для открытия пор образец подвергали химическому травлению в растворе, содержащем $20\text{г/л}$ $CrO_3$ и $35\text{мг/л}$ $H_3PO_4$ при температуре $60^{\circ}С$ в течение $15$ минут с постоянным перемешиванием.

\subsubsection{Стравливание алюминия}

Для стравливания алюминия использовался раствор $CuCl_2$ и $HCl$. Образец выдерживался в растворе до полного растворения алюминия. Полученную оксидную плёнку обильно промывали дистиллированной водой после чего высушивали в эксикаторе.

\subsection{Методика магнетронного напыления Ni}

В работе для получения наноразмерных слоёв $Ni$ на поверхности ПОА использовали метод магнетронного распыления. Один из возможных вариантов схем магнетронного распылителя представлен на рис. \ref{magnetron}.

\addimg{magnetron}{0.45}{Схема установки для магнетронного распыления \cite{vacuumnoe}.}{magnetron}

Цифрами обозначены:

\newlist{list1}{enumerate}{10}
\setlist[list1]{label*=\arabic*.}
\setlistdepth{1}
\begin{list1}
\item мишень, одновременно являющаяся катодом распылительной системы;
\item постоянный магнит, создающий магнитное поле, силовые линии которого параллельны поверхности мишени;
\item кольцевой анод.
\end{list1}

Выше анода располагается подложка (на рисунке не показана), на которой формируется пленка из материала мишени.

Особенность магнетронного распылителя заключается в наличии двух скрещенных полей – электрического и магнитного.
Траектория движения электрона, испущенного из мишени-катода (за счет вторичной электронной эмиссии), будет определяться действием на него этих полей. Под воздействием электрического поля электрон начнет двигаться к аноду. Действие магнитного поля на движущийся заряд приведет к возникновению силы Лоренца, направленной перпендикулярно скорости. Суммарное действие этих сил приведет к тому, что в результате электрон будет двигаться параллельно поверхности мишени по сложной замкнутой траектории, близкой к циклоиде.
Здесь важно то, что траектория движения является замкнутой. Электрон будут двигаться по ней до тех пор, пока не произойдет несколько столкновений его с атомами рабочего газа, в результате которых произойдет их ионизация, а сам электрон, потеряв скорость, переместится за счет диффузии к аноду. Замкнутый характер траектории движения электрона резко увеличивает вероятность его столкновения с атомами рабочего газа. Это означает, что газоразрядная плазма может образовываться при более низких давлениях. Следовательно и пленки, полученные методом магнетронного распыления, будут более чистыми. Другое преимущество магнетронных систем заключается в том, что ионизация газа происходит непосредственно вблизи поверхности мишени. Газоразрядная плазма локализована вблизи мишени, а не "размазана" в межэлектродном пространстве. Как следствие - резко возрастает интенсивность бомбардировки мишени ионами рабочего газа, тем самым увеличивается скорость распыления мишени и, следовательно, скорость роста пленки на подложке (скорость достигает несколько десятков нм/с). Наличие магнитного поля не дает электронам, обладающим высокой скоростью, долететь до подложки, не столкнувшись с атомами рабочего газа. Поэтому подложка не нагревается вследствие бомбардировки ее вторичными электронами. Основным источником нагрева подложки является энергия, выделяемая при торможении и конденсации осаждаемых атомов вещества мишени, в результате чего температура подложки не превышает $100 - 200^\circ C$. Это дает возможность напылять пленки на подложки из материалов с малой термостойкостью (пластики, полимеры, оргстекло и так далее) \cite{vacuumnoe}. 
Процесс напыления происходил в камере, вакуумированной до давления $10^{−5}\text{Торр}$, на установке Angstrom Engineering Inc. Nexdep Series. Скорость распыления составляла $4\text{Å}/\text{с}$, мощность магнетрона − $70\%$, ток – $0,4\text{А}$, напряжение – $52\text{В}$. С учётом известной скорости ($4 \text{Å}/\text{с}$) распыления мишени рассчитывалось приблизительное время самого процесса распыления. Время напыления слоя заданной толщины $50\text{нм}$ составило приблизительно $2$ минуты. Толщина была определена исходя из среднего диаметра поры ($\sim50\text{нм}$). При увеличении толщины слоя $Ni$ наблюдается сплошное зарастание поры. Режим натекания аргона устанавливался такой, что его стационарное давление составляло $10^{-3}\text{Торр}$, остаточное давление в камере – $10^{-6}\text{Торр}$.

\subsection{Методы исследования}

Исследование морфологии поверхности осуществлялось методами оптической, атомно-силовой, и растровой электронной микроскопии.

\subsubsection{Оптическая микроскопия}

Метод оптической микроскопии использовался для определения качества поверхности образцов, прошедших электрохимическую полировку, а также для выбора оптимальной области для посадки зонда (на участке не должно быть видимых перепадов высот, неровностей и т.д.). Зачастую на алюминиевой фольге невооружённым глазом не видно царапин, следовательно, для выявления степени этих повреждений, и пригодности фольг к последующему анодному оксидированию необходим предварительный визуальный анализ посредством оптического микроскопа. Использовался микроскоп NanoHardness Tester (CSM Instruments).

\subsubsection{Атомно-силовая микроскопия}

Метод атомно-силовой микроскопии (АСМ) может использоваться для исследования не только проводящих поверхностей, но и поверхностей полупроводников, и даже и изоляторов \cite{kelcall}, что является весьма актуальным в настоящей работе в связи с особенностью синтезируемого пористого оксида алюминия. Помимо непосредственного исследования топографии поверхности методом контактной атомно-силовой микроскопии, АСМ позволяет регистрировать силы трения, магнитные, электростатические и адгезионные силы, распределения поверхностного потенциала и электрической ёмкости и т.д. \cite{eliseev-functsionalnie}.

В рамках данной работы методом АСМ проводили исследование поверхности алюминиевой фольги после проведения электрохимического полирования, а также двухстадийного анодного оксидирования алюминия с целью изучения пористой структуры полученного оксида. Рассматривались такие параметры, как наличие пор, диаметр и их взаимное расположение. Сканирование осуществлялось с помощью атомно – силового микроскопа Solver P47 (рисунок 18) в полуконтактном режиме.

\addimg{solver}{0.75}{Атомно - силовой микроскоп Solver P47.}{solver}

\subsubsection{Растровая элетронная микроскопия}

Растровая электронная микроскопия (РЭМ) является чрезвычайно информативной при визуализации микроструктуры поверхности и приповерхностной зоны \cite{kelcall}. Методом растровой электронной микроскопии можно исследовать микрорельеф, распределение химического состава и электронной плотности и т.д. \cite{eliseev-functsionalnie}. Метод РЭМ использовали для исследования морфологии скола синтезированных методом анодного оксидирования плёнок пористого оксида алюминия после магнетронного напыления $Ni$ с помощью микрофотографий, полученных микроскопом JEOL JSM 6510 LV.

\clearpage
