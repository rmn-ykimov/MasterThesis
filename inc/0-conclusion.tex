\anonsection{Заключение}

В ходе выполнения выпускной квалификационной работы магистра была изучена научно-техническая литература по заявленной тематике, основываясь на которой были выявлены основные подходы к созданию нанокомпозитов $Ni$/ПОА (метод электрохимического осаждения и метод магнетронного распыления), также была изучена методика вакуумного магнетронного распыления, освоен мат. аппарат для расчета диаметра пор, количества нанонитей в порах и их размеров. На основании экспериментальных и расчетных данных из работ, представленных в литературном обзоре, были выявлены оптимальные параметры методики синтеза $Ni$/ПОА (методика двустадийного двустороннего анодирования с последующим магнетронным распылением $Ni$).

В рамках экспериментального исследования были отработаны режимы анодного оксидирования и синтез матриц пористого оксида алюминия (ПОА) с массивом упорядоченных открытых пор, а также было произведено осаждение $Ni$ методом магнетронного напыления в поры оксида алюминия. Интегральная толщина слоя $Ni$ составила $50\text{нм}$. На основании данных микроскопии (различные методы АСМ, РЭМ), полученных в ходе экспериментальных исследований, был произведен анализ итоговой гетероструктуры. Исходя из этого анализа был сделан вывод о присутствии $Ni$ как в объёме пор ПОА, так и на поверхности.

Практическая значимость магистерской работы состоит в выявлении оптимальной методики синтеза ПОА, а также в возможном удешевлении стоимости производства гетероструктур на его основе.

Таким образом, задачи исследовательской работы выполнены в полном объеме.

\clearpage
